\documentclass[12pt,a4paper]{article}
\usepackage{geometry}
\usepackage{titlesec}
\usepackage{hyperref}
\geometry{margin=1in}
\usepackage{graphicx}
\usepackage{fancyhdr}

\titleformat{\section}{\normalfont\Large\bfseries}{\thesection}{1em}{}
\titleformat{\subsection}{\normalfont\large\bfseries}{\thesubsection}{1em}{}

\begin{document}

\begin{titlepage}
\begin{center}
\vspace{2cm}
%\textsc{Oregon State University}\\[1.5cm]
\includegraphics[width=1\textwidth]{dpsSociety.jpg}\\[1cm]
\vspace{2cm}
% Title
\hrule
\vspace{.5cm}
{ \huge \bfseries COMPUTER SCIENCE PROJECT} % title of the report
\vspace{.5cm}
\hrule
\vspace{1.5cm}
\textsc{\textbf{Author}}\\
\vspace{.5cm}
\centering
% add your name here
Name-NIKET BASU\\
Cls-XII J ; Roll No:17\\
Adm No:14184/2024\\
\vspace{4cm}

\end{center}
\end{titlepage}

\author{Prepared by: Niket Basu}
\date{}
\title{{\bfseries Quiztastic}}

% Set up fancy page style BEFORE \maketitle
\pagestyle{fancy}
\fancyhf{} % Clear all headers and footers
\rhead{\nouppercase{\rightmark}}
\lhead{Computer Science Synopsis} % Use actual title text instead of \title command
\rfoot{\includegraphics[height=1cm]{logo.png}} % right footer logo
\cfoot{\thepage}
\setlength\headheight{16pt}
\setlength{\footskip}{50pt}

% Also define the plain style to match fancy style for first page
\fancypagestyle{plain}{%
  \fancyhf{} % Clear all headers and footers
  \rhead{\nouppercase{\rightmark}}
  \lhead{Computer Sceince Synopsis}
  \rfoot{\includegraphics[height=1cm]{logo.png}}
  \cfoot{\thepage}
}

\maketitle


\section{Objective of the Proposed System}

The objective of the ``Quiztastic'' project is to develop an interactive quiz platform that enables users to test and expand their knowledge across various subjects and difficulty levels. The system is designed to provide a seamless experience for users to register, log in, participate in quizzes, and view their performance on a leaderboard. By integrating with an external trivia API, the application offers a diverse range of questions while maintaining secure user authentication and persistent scoring via a database. This project aims to foster self-paced learning, healthy competition, and engagement through gamification.

\textbf{Key Objectives:}
\begin{itemize}
    \item Automate the quiz-taking process with a user-friendly GUI.
    \item Ensure secure registration and login for users.
    \item Offer diverse quiz categories and difficulty levels.
    \item Store and update user scores in a persistent database.
    \item Motivate users through real-time feedback and leaderboards.
    \item Provide a scalable foundation for future educational enhancements.
\end{itemize}

\section{Input and Output of the Proposed System}

\subsection*{Inputs}
\begin{itemize}
    \item \textbf{User Details:} Username and password for registration and authentication.
    \item \textbf{Quiz Preferences:} Selection of quiz category, difficulty level, question type, and number of questions.
    \item \textbf{Quiz Responses:} User-selected answers to each quiz question.
\end{itemize}

\subsection*{Outputs}
\begin{itemize}
    \item \textbf{Quiz Questions:} Dynamically fetched from the Open Trivia DB API based on user preferences.
    \item \textbf{Immediate Feedback:} Correctness of answers after each submission.
    \item \textbf{Score Display:} User's score upon quiz completion.
    \item \textbf{Leaderboard:} Ranked list of top users based on cumulative scores.
    \item \textbf{Registration/Login Feedback:} Success or error messages for authentication.
\end{itemize}

\subsection*{Data Flow Example}
\begin{enumerate}
    \item User enters login credentials $\rightarrow$ System validates using database.
    \item User selects quiz settings $\rightarrow$ System fetches relevant questions from API.
    \item User answers questions $\rightarrow$ System evaluates and provides feedback.
    \item Upon completion, total score is updated and displayed.
    \item User can view the leaderboard at any time.
\end{enumerate}

\section{Functions or Features of Proposed System}

\textbf{Major Features:}
\begin{itemize}
    \item \textbf{User Registration \& Login:}
    \begin{itemize}
        \item Secure sign-up with hashed password storage.
        \item Login authentication against SQLite database.
    \end{itemize}
    \item \textbf{Quiz Generation:}
    \begin{itemize}
        \item Fetches categories and questions from Open Trivia DB.
        \item Customizable quiz settings (category, difficulty, type, amount).
    \end{itemize}
    \item \textbf{Quiz Participation:}
    \begin{itemize}
        \item Presents questions one at a time with multiple choice/boolean options.
        \item Provides immediate feedback on each answer.
    \end{itemize}
    \item \textbf{Scoring System:}
    \begin{itemize}
        \item Tracks and updates user's cumulative score in database.
        \item Displays score at quiz end.
    \end{itemize}
    \item \textbf{Leaderboard:}
    \begin{itemize}
        \item Shows top 10 users ranked by score.
        \item Motivates with real-time competitive statistics.
    \end{itemize}
    \item \textbf{GUI Navigation:}
    \begin{itemize}
        \item Intuitive navigation between login, home, quiz, and leaderboard screens.
    \end{itemize}
    \item \textbf{Error Handling \& User Guidance:}
    \begin{itemize}
        \item Informative messages for invalid inputs, API errors, and quiz completion.
    \end{itemize}
\end{itemize}

\textbf{Additional Features for Future Expansion:}
\begin{itemize}
    \item Biometric authentication.
    \item Quiz history and analytics.
    \item Group quizzes and challenges.
\end{itemize}

\section{Front-end and Back-end to be Used}

\subsection*{Front-end}
\begin{itemize}
    \item \textbf{Technology:} Python Tkinter (GUI library)
    \item \textbf{Components:}
    \begin{itemize}
        \item Frames for each major screen (Login, Home, Quiz, Leaderboard)
        \item Widgets: Labels, Buttons, Entry fields, Radiobuttons, Comboboxes, Spinboxes
        \item Styling for improved user experience (fonts, colors, layouts)
    \end{itemize}
\end{itemize}

\subsection*{Back-end}
\begin{itemize}
    \item \textbf{Database:} SQLite3
    \begin{itemize}
        \item Stores user credentials and scores
        \item Handles queries for registration, authentication, scoring, and leaderboard
    \end{itemize}
    \item \textbf{API Interaction:} Open Trivia DB
    \begin{itemize}
        \item Fetches quiz categories and questions remotely
    \end{itemize}
    \item \textbf{Other Libraries:}
    \begin{itemize}
        \item \texttt{requests} for making HTTP requests to API.
        \item \texttt{hashlib} for password hashing.
        \item \texttt{random} for shuffling answer options.
        \item \texttt{html} for handling special characters in questions and answers.
    \end{itemize}
\end{itemize}

\textbf{Architecture Overview:}\\
The GUI interacts with the back-end database and the trivia API, ensuring smooth data flow and real-time user feedback.

\section{Hardware \& Software to be Used}

\subsection*{Hardware Requirements}
\begin{itemize}
    \item Standard personal computer or laptop
    \item Minimum 2GB RAM, recommended 4GB+
    \item No specialized hardware required (optional biometric device for future)
\end{itemize}

\subsection*{Software Requirements}
\begin{itemize}
    \item \textbf{Operating System:} Windows/Linux/Mac OS
    \item \textbf{Programming Language:} Python 3.x
    \item \textbf{Libraries:}
    \begin{itemize}
        \item Tkinter (built-in with Python)
        \item requests (\texttt{pip install requests})
        \item SQLite3 (built-in with Python)
        \item hashlib, random, html (standard library)
    \end{itemize}
    \item \textbf{Additional Tools:}
    \begin{itemize}
        \item Internet connection (for API access and initial setup)
        \item IDE/Text Editor (e.g., Visual Studio Code, PyCharm, IDLE)
    \end{itemize}
\end{itemize}

\section{Scope and Limitations of the Project}

\subsection*{Scope}
\begin{itemize}
    \item Provides an engaging platform for individual users to participate in quizzes.
    \item Suitable for educational institutions, self-learners, and competitive environments.
    \item Easily extendable for new features (group quizzes, analytics, additional question sources)
    \item Promotes gamification and learning retention.
\end{itemize}

\subsection*{Limitations}
\begin{itemize}
    \item \textbf{Single-device focus:} No native support for mobile/web platforms.
    \item \textbf{API dependency:} Quiz content relies on external API availability and quality.
    \item \textbf{Database scalability:} SQLite is ideal for small to moderate user bases but not for large-scale deployments.
    \item \textbf{Limited user analytics:} No detailed tracking of user progress or question statistics.
    \item \textbf{No offline quiz:} Requires internet for fetching new questions.
\end{itemize}


\section*{Conclusion}

\noindent
\textit{Quiztastic} is a robust and scalable quiz platform that leverages Python, Tkinter, and SQLite to deliver a dynamic and interactive user experience. The system supports secure user management, customizable quizzes, real-time scoring, and competitive leaderboards. With clear design and modular architecture, it offers a foundation for future enhancement and adaptation to broader educational needs.




\end{document}